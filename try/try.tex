\documentclass{ransom}

\begin{document}

\begin{raggedright}\begin{multicols}{2}
\common\\
\vocab{ἀνὰ}{upon (+gen), up}\\
\vocab{ἄναξ}{lord}\\
\vocab{βουλή}{will, plan}\\
\vocabinflection{ἔθηκε}{τίθημι}{put, put in a state}\\
\vocab{ἔρις}{strife}\\
\vocab{ἥρως}{hero}\\
\vocab{θεά}{goddess}\\
\vocab{θυγάτηρ}{daughter}\\
\vocabinflection{κύνεσσιν}{κύων}{dog}\\
\vocab{λαός}{people, army}\\
\vocab{λίσσομαι}{beg, beseech}\\
\vocab{λύσομαι}{free, ransom}\\
\vocab{μάλιστα}{greatly}\\
\vocab{στρατὸν}{army}\\
\vocab{σφωε}{they (dual)}\\
\vocab{τελέω}{accomplish}\\
\vocab{τεύχω}{make}\\
\vocabinflection{χερσὶν}{χείρ}{hand}\\
\vocab{χρύσεος}{golden}\\
\vocabinflection{ὦρσε}{ὄρνυμι}{stir up}\\

\bigseparator

\uncommon\\
\vocab{ἀείδω}{sing}\\
\vocab{ἄλγος}{woe}\\
\vocab{ἀπερείσιος}{boundless}\\
\vocab{ἄποινα}{ransoms}\\
\vocab{ἀρητήρ}{priest}\\
\vocab{ἀτιμάζω}{insult, dishonor}\\
\vocab{βασιλεύς}{king}\\
\vocabinflection{διαστήτην}{διίστημι}{separate}\\
\vocab{ἑκηβόλος}{sharpshooter}\\
\vocab{ἑλώριον}{booty}\\
\vocab{ἐρίζω}{struggle}\\
\vocab{θοός}{swift}\\
\vocab{ἴφθιμος}{strong}\\
\vocab{κοσμήτωρ}{commander}\\
\vocab{μῆνις}{rage}\\
\vocab{μυρίος}{immense, myriad}\\
\vocab{νόσος}{disease}\\
\vocabinflection{ξυνέηκε}{συνίημι}{bring together}\\
\vocab{οἰωνός}{vulture, omen}\\
\vocab{ὀλέκω}{destroy, kill}\\
\vocab{οὐλόμενος}{ruinous}\\
\vocab{οὕνεκα}{because}\\
\vocab{σκῆπτρον scepter}\\
\vocab{προϊάπτω}{throw}\\
\vocab{στέμμα}{band, ribbon}\\
\vocab{χολόω}{anger, provoke}\\
\vocab{ψυχή}{soul}\\

\end{multicols}\end{raggedright}

\pagebreak

\begin{greek}\begin{raggedright}\begin{doublespace} \large

Μῆνιν ἄειδε, θεά, Πηληϊάδεω Ἀχιλῆος \\
οὐλομένην, ἣ μυρί᾽ Ἀχαιοῖς ἄλγε᾽ ἔθηκε, \\
πολλὰς δ᾽ ἰφθίμους ψυχὰς Ἄϊδι προΐαψεν \\
ἡρώων, αὐτοὺς δὲ ἑλώρια τεῦχε κύνεσσιν \\
οἰωνοῖσί τε πᾶσι· Διὸς δ᾽ ἐτελείετο βουλή· \\
ἐξ οὗ δὴ τὰ πρῶτα διαστήτην ἐρίσαντε \\
Ἀτρεΐδης τε ἄναξ ἀνδρῶν καὶ δῖος Ἀχιλλεύς. \\
Τίς γάρ σφωε θεῶν ἔριδι ξυνέηκε μάχεσθαι; \\
Λητοῦς καὶ Διὸς υἱός· ὁ γὰρ βασιλῆϊ χολωθεὶς \\
νοῦσον ἀνὰ στρατὸν ὦρσε κακήν, ὀλέκοντο δὲ λαοί, \\
οὕνεκα τὸν Χρύσην ἠτίμασεν ἀρητῆρα \\
Ἀτρεΐδης· ὃ γὰρ ἦλθε θοὰς ἐπὶ νῆας Ἀχαιῶν \\
λυσόμενός τε θύγατρα φέρων τ᾽ ἀπερείσι᾽ ἄποινα, \\
στέμματ᾽ ἔχων ἐν χερσὶν ἑκηβόλου Ἀπόλλωνος \\
χρυσέῳ ἀνὰ σκήπτρῳ, καὶ λίσσετο πάντας Ἀχαιούς, \\
Ἀτρεΐδα δὲ μάλιστα δύω, κοσμήτορε λαῶν·

\end{doublespace}\end{raggedright}\end{greek}

\pagebreak

Sing, Ο goddess, the destructive wrath of Achilles, son of Peleus,
which brought countless woes upon the Greeks, 1 and hurled many valiant
souls of heroes down to Hades, and made themselves 2 a prey to dogs and
to all birds [but the will of Jove was being accomplished], from the
time when Atrides, king of men, and noble Achilles, first contending,
were disunited.

Which, then, of the gods, engaged these two in strife, so that they
should fight? 3 The son of Latona and Jove; for he, enraged with the
king, stirred up an evil pestilence through the army [and the people
kept perishing] 4; because the son of Atreus had dishonoured the priest
Chryses: for he came to the swift ships of the Greeks to ransom his
daughter, and bringing invaluable ransoms, having in his hands the
fillets of far-darting Apollo on his golden sceptre. And he supplicated
all the Greeks, but chiefly the two sons of Atreus, the leaders of the
people:


\end{document}
