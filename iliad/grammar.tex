\pagebreak

\begin{small}

\newcommand{\grammartablehdr}[1]{{\textcolor{gray}{\emph{#1}}}}

\mychapter{Field guide to Homeric grammar}

\formatlikesection{Pronouns}

Homer has about 209 pronouns:

{\footnotesize 
ἅ αἱ αἵ αἵδε ἄμμε ἄμμες ἄμμι ἄμμιν ἅς ἄσσα ἅσσα ἑ ἕ ἐγώ ἔγωγε ἐγών ἑέ ἕης ἑθέν ἑθεν ἕθεν εἷο ἐμέ ἐμέθεν ἐμεῖο ἐμέο ἐμεῦ ἐμοί ἔμοιγε ἑο ἕο ἑοῖ εὑ ἥ ἥδε ἧμας ἡμέας ἡμεῖς ἡμείων ἡμέων ἡμῖν ἥμιν ἧμιν ἥν ἧς ᾗς κεῖνος μέ με μευ μίν μιν μοί μοι νώ νῶΐ νῶι νῶϊ νῶιν νῶϊν ὅ ὅδε οἱ οἵ οἷ οἵδε οἷσί οἷσι οἷσίν οἷσιν ὅν ὅου ὅς ὅστις ὁτέοισιν ὅτευ ὅτεῳ ὅτινα ὅτινας ὅτις ὅττευ οὗ οὕς σέ σε σέθεν σεῖο σέο σεο σευ σεῦ σοί σοι σοῦ σύ σφας σφε σφέας σφεας σφείων σφέων σφεων σφι σφιν σφίσι σφισι σφίσιν σφισιν σφώ σφωε σφῶΐ σφῶϊ σφωιν σφωϊν σφῶϊν σφῶν σφῷν τά τάδε ταί τάς τάσδε τάων τεΐν τέο τεο τεοῖο τευ τεῦ τεῳ τέων τῇδέ τήν τήνδε τῆς τῇς τῆσδέ τῆσδε τῇσι τῇσίν τῇσιν τί τι τίνα τινά τινα τινάς τινας τινε τίνες τινές τινες τινι τίς τις τό τόδε τοί τοι τοιάδε τοιαίδε τοιήδε τοῖιν τοῖϊν τοῖο τοιοίδε τοιόνδε τοιόσδε τοιούσδε τοῖσδε τοίσδεσι τοίσδεσσι τοῖσδεσσι τοίσδεσσιν τοῖσί τοῖσι τοῖσίν τοῖσιν τόν τόνδε τοσόνδε τοσσάδε τοσσόνδε τοῦ τοῦδέ τοῦδε τούς τούσδε τύνη τώ τῳ τῷ τώδε τῷδε τῶν τῶνδε ὑμέας ὑμεῖς ὑμείων ὑμῖν ὕμιν ὔμμε ὔμμες ὔμμι ὔμμιν χἠμεῖς ὥ ᾧ ὧν
}

They proliferate because (1) Homer mixes Aeolic and Ionian words, (2) some pronouns come in both emphatic and unemphatic forms, and
(3) sometimes there are contractions of ε.

\formatlikesubsection{Personal pronouns}

The most common Ionian personal pronouns are:\\
%
\begin{tabular}{lll}
N & ἐγώ σύ εἷο     & ἡμεῖς ὑμεῖς -- \\
G & ἐμεῖο σεῖο εἷο & ἡμείων ὑμείων σφείων \\
D & ἐμοί σοί ἑοί   & ἡμῖν ὑμῖν σφίσι \\
A & ἐμέ σέ ἑέ      & ἡμέας ὑμέας σφέας
\end{tabular}\\
%
These forms are used for emphasis and with prepositions.
Contractions happen mainly in the genitive. They take -εῖο to -έο and -εῦ (both occur), and -είων to -έων.
A few other contractions exist, such as ἕ=ἑέ and σφάς=σφέας. Τεΐν=σοί. 

The older Aeolic forms that differ are:\\
%
\begin{tabular}{lll}
N & -- -- --  & ἄμμες ὔμμες -- \\
G & ἔμεθεν σέθεν ἕθεν & -- -- -- \\
D & -- -- --  & ἄμμιν ὔμμιν -- \\
A & -- -- --  & ἄμμε ὔμμε -- 
\end{tabular}

The third-person pronouns, where they exist, are actually not personal but
rather refer to other words or phrases, although like the true personal
pronouns they are not inflected for gender. Sometimes they are used as reflexives.
They are uncommon in Homer, and more
frequently he uses forms of ὁ, ἡ, τό, which can be used for this purpose as well
as being demonstrative and relative pronouns. Example: τὴν δ᾽ ἐγὼ οὐ λύσω,
``but I will not release her'' (Iliad 1.29).

The unemphatic forms are:\\
%
\begin{tabular}{lll}
G & μευ σεο+σευ ἑο+ἑυ & -- -- σφεων \\
D & μοι τοι ὁι        & -- -- σφισι \\
A & με σε ἑ+μιν       & -- -- σφεας
\end{tabular}\\
%
These are enclitic. The distinction between emphatic and unemphatic pronouns
is usually reinforced by word order:
``δοκεῖ μοι,'' but ``ἐμοὶ δοκεῖ.'' As in English, the pronoun's normal position is after
the verb (the dog bit me), and fronting it is for emphasis (it's me
that the dog bit). For stronger emphasis, -γε can be added: ἔγωγε.
% Opposite of French, e.g., dites moi, me dire, because in french normal position is before verb.
% But not always?
% See Dik, ``On Unemphatic `Emphatic' Pronouns in Greek,'' sci-hub.se/10.2307/4433482
% Her example: Μή σε, γέρον κοίλῃσιν ἐγὼ παρὰ νηυσὶ κιχείω (Iliad 1.26).

Duals are formed with νῶι- (1p) and σφωι- (2+3p), and end in -ν for the genitive and dative.

\formatlikesubsection{Correlatives}

\begin{tabular}{llllll}
& \grammartablehdr{interrog.} 	& \grammartablehdr{some+X} & 	\grammartablehdr{demonstr.} & 	\grammartablehdr{rel.} & 	\grammartablehdr{X+ever} \\
& τίς &    τις &     &           ὅς &     ὅστις  \\
& ποῦ &    που &         &        οὗ &     ὅπου \\
& πότε &   ποτέ &    τότε &       ὅτε &    ὁπότε \\
& πῶς &    πως &     οὕτως &      ὡς &     ὅπως \\
& ποῖος &  ποιός &   τοιοῦτος &   οἷος &   ὁποῖος \\
& πόσος &  ποσός &   τοσοῦτος &   ὅσος &   ὁπόσος
\end{tabular}

\pagebreak

\formatlikesection{Itty bitties}

What I mean by an ``itty bitty'' is a short word that contributes disproportionately to confusion.
Many of the following words are two or three letters, and many are among the ten or twenty the most common
words in the Homeric dialect. Many are verbal particles, clitics, or postpositives.

A \emph{particle} is a word that has no meaning of its own and changes the meaning of other words.
A \emph{clitic} is a word that gets controlled phonologically by other words, leaning (κλίνω) on them. In Greek,
clitics lack an accent, although they are listed with one in dictionary entries. 
They consist of \emph{proclitics} that lean on the following word, and
\emph{enclitics} that lean on the preceding word. Articles and prepositions are enclitics.
A \emph{postpositive} is a word that comes after the word that it modifies, as in ``someone nice.''



\formatlikesubsection{Conjunctions}

καί - and\\
ἀλλά - but\\
γάρ - for; postpositive\\
δέ - and, but (not a negative like modern δεν)\\
ὅμως - nevertheless\\

\formatlikesubsection{Negation}

οὐ(κ), οὐχ - not; proclitic\\
μή - negative form used in imperatives\\

\formatlikesubsection{List of itty bitties}


κε(ν) - used like ``if'' to limit verbs\\
αν - like κε; in Homer, may be more emphatic or used more often for negative clauses\\
εἰ - ``if;'' can be used, e.g., as εἰ κεν + verb; proclitic\\
αἰ - Aeolic form of εἰ; may imply a wish or purpose\\
αἰ κε(ν) - if only, so that\\
αἲ γάρ - oh, that \ldots !\\
ἤν = εἰ ἄν; also an interjection, ``see there!;'' cf.~epic pronoun ἥν\\
γε - used before or after a word to mark or emphasize it; often ``at least;'' postpositive, enclitic\\
ἄρ(α)/ῥα - time or causation: then/next, therefore; postpositive; in later dialects, can introduce
      a question, as in ``Who, then, will fight?''\\
γάρ = γε ἄρ ``for;'' postpositive\\
δέ - but, and, or supplying the reason for something; cf. postposition -δε, ``to''\\
αὐτάρ, ἀτάρ - similar to δέ, poetic\\
τε - correlative/connecting particle; always postpositive?; enclitic\\
μέν, μήν - affirmative particles; the difference depends on prose/verse and meter\\
τοι - (1) synonym for dative pronoun σοι; (2) affirmative particle; both enclitic\\

\formatlikesubsection{Oaths, emphasis, and emphatics}

δή - indeed, truly; postpositive\\
ἦ μέν - used in oaths\\
ἤτοι - indeed, truly (also used in either/or constructions)\\

\formatlikesubsection{Time, causation, and temporal order}

ἤδη - already, now\\
νῦ(ν) - adverb; now, just now, presently; cf.~enclitic νυν (rare in Homer)\\
οὖν - postpositive adverb; so, then\\
ἄρ(α)/ῥα - so, then, after all\\
ὥστε - so that; adverb+inf; conjuction\\
ἅμα - at the same time with, together with\\
εὖτε - when, as, since\\
τέως/ἕως - meanwhile, for a time\\
ἵνα - so that\\

\formatlikesubsection{Correlatives}

In later dialects, where articles are common, one often has
the postpositive between the article and the noun, e.g.,
ὁ τ᾽ ἥλιος καὶ τὴν σελήνη.

τε \ldots και - A τε \ldots και Β, both A and B\\
μέν \ldots δέ \ldots - contrasting, ``and on the other hand''\\
οὔτε \ldots οὔτε - neither \ldots nor\\

\formatlikesection{Verbs}

Active, middle, active infinitive, and active participles of a thematic verb:

{\footnotesize

\begin{tabular}{ll}
\grammartablehdr{present (1ο)}\\
     διδάσκω διδάσκεις διδάσκει        & διδάσκομεν διδάσκετε διδάσκουσιν \\
     διδάσκομαι διδάσκεαι διδάσκεται   & διδασκόμεθα διδάσκεσθε διδάσκονται \\
     διδάσκειν \\
     διδάσκων διδάσκουσα διδάσκον \\
\grammartablehdr{imperfect (+1ο)}\\
     ἐδίδασκον ἐδίδασκες ἐδίδασκε      & ἐδιδάσκομεν ἐδιδάσκετε ἐδίδασκον \\
     ἐδιδασκόμην ἐδιδάσκεω ἐδιδάσκετο  & ἐδιδασκόμεθα ἐδιδάσκεσθε ἐδιδάσκοντο \\
\grammartablehdr{future (1σο)}\\
     διδάξω  διδάξεις  διδάξει         & διδάξομεν διδάξετε διδάξουσιν \\
     διδάξομαι διδάξεαι διδάξεται      &  διδαξόμεθα  διδάξεσθε  διδάξονται \\
     διδάξειν \\
     δῐδᾰ́ξων δῐδᾰ́ξουσᾰ δῐδᾰ́ξον \\
\grammartablehdr{aorist (+1σα)}\\
     ἐδίδαξα  ἐδίδαξας ἐδίδαξε         & ἐδιδάξαμεν  ἐδιδάξατε ἐδίδαξαν \\
     ἐδιδαξάμην      ἐδιδάξω         ἐδιδάξατο       & ἐδιδάξασθον   ἐδιδαξάσθην     ἐδιδαξάμεθα     ἐδιδάξασθε      ἐδιδάξαντο \\
     αδιδάξαι \\
     διδάξας διδάξασα διδάξαν \\
\end{tabular}

}

\pagebreak

\newcommand{\tca}{\cellcolor{TableColorA}}
\newcommand{\tcb}{\cellcolor{TableColorB}}

Infixes in verbs:\\*
%
\begin{tabular}{llll}
root aor.         & η/ω/υ & passive aor.      & σθη \\
future              & σ     & future passive      & θεσ \\
perfect active      & κ     & future middle perfect & κσ \\
present, 2nd aor. opt.    & οι(η) (αοι->ῳ) & participle        & ντ\footnotemark \\
aor. optative     & \multicolumn{3}{l}{(σ)αι(η)/(σ)ει(η)}\\
\end{tabular}

Personal endings:\\*
%
\begin{tabular}{lllllll}
\textbf{active, -ω verbs}\\
present, future              & ω      & εις\footnotemark  & ει\footnotemark[3]  & ομεν    & ετε    & ουσι(ν) \\
impf.; them. 2nd aor.        & ον\footnotemark[4] & ες      & ε(ν)      & ομεν    &	ετε    & ον\footnotemark[4] \\
aorist                       & α      & ας      & ε(ν)      & αμεν    & ατε    & αν\footnotemark[5] \\
root aorist (-η/ω/υ-)        & \tca{}ν & \tca{}ς  & \tca{}-  & \tca{}μεν & \tca{}τε & \tca{}σαν \\
\textbf{active, -μι verbs}\\
present                      & μι     & ς       & σι        & μεν     & τε     & ασι \\
aorist, impf.                & \tca{}ν & \tca{}ς  & \tca{}-  & \tca{}μεν & \tca{}τε & \tca{}σαν \\
\textbf{middle}\\
present                      & \tcb{}μαι & σαι  & \tcb{}ται & \tcb{}μεθα & \tcb{}σθε & \tcb{}νται \\
impf.; aor. ind., opt.       & μην    & σο      & το        & μεθα    &	σθε    & ντο/ατο\\
\textbf{passive}\\
fut; pres., athematic        & \tcb{}μαι  & αι  & \tcb{}ται & \tcb{}μεθα & \tcb{}σθε & \tcb{}νται \\
\ldots thematic              & ομαι   & ει\footnotemark[3] & εται      & ομεθα   & εσθε   & ονται \\
aor. (1st θη-, 2d η-)        & \tca{}ν & \tca{}ς  & \tca{}-/το  & \tca{}μεν & \tca{}τε & \tca{}σαν/θεν/ντο \\
\textbf{optative}\\
present                      & μι/ην  & ς/ης    & -/η/οῖ    & μεν     & τε     & εν \\
aorist                       & μι     & ς       & -         & μεν     & τε     & εν \\
aorist, alt. forms           &        & ας      & ε(ν)      &         &        & αν \\
\textbf{imperative}\\
active                       &        & ε/θι/τι\footnotemark[6] & \textcolor{gray}{τω}        &         & τε     & \textcolor{gray}{ντων} \\
mediopassive                 &        & σο/ου      & \textcolor{gray}{σθω}       &         & σθε    & \textcolor{gray}{σθων} \\
\end{tabular}

\footnotetext[1]{also 3 pl mp}
\footnotetext[2]{also participle}
\footnotetext[3]{both active 3 and pass 2}
\footnotetext[4]{both 1 sing and 3 pl}
\footnotetext[5]{both aorist and some participles}
\footnotetext[6]{θι is omitted in present, and becomes τι after another aspirated consonant}

\pagebreak

% Smyth, sec. 469
% https://latin.stackexchange.com/a/16856/3597

Infinitives:\\*
%
\begin{tabular}{p{1.25in}p{2.75in}}
-εν/ειν, -μεναι, -μεν   &  present, thematic 2nd aor active, fut active \\
-αι\footnotemark[7]    &  1st aor active \\
-ναι, -μεναι, -μεν  & present and 2nd perfect of athematic verbs; passive aor; perfect active \\
-μεναι              &  athematic 2nd aor \\
-σθαι               &  mediopassive, except aor passive \\
\end{tabular}

Participles:\\*
%
\begin{tabular}{lllll}
                          & \multicolumn{4}{l}{[\emph{-αντο is middle impf., never a participle.}]} \\
\multicolumn{5}{l}{\textbf{present, 2nd aorist}}\\
masc.~($\sim$ γέρων)      & ων\footnotemark[8]/ους & οντος  & οντι  & οντα      \\
                          & οντες  & οντων & ουσι(ν) & οντας \\
fem.~($\sim$ θάλασσα)     & ουσα   & ουσης & ούσῃ & ουσαν \\
                          & ουσαι  & ουσάων & ούσῃς & ουσας \\
neut.                     & ον     & οντος  & οντι  & ον \\
                          & οντα   & οντων  & ουσι(ν) & οντα \\
\multicolumn{5}{l}{\textbf{middle}}\\
                          & μενος \\
\multicolumn{5}{l}{\textbf{1st aorist}}\\
masc.~($\sim$ πᾶς)        & ας     & αντος  &αντι     & αντα \\
                          & αντες  & αντων  & ασι(ν)  & αντας \\
fem.~($\sim$ πᾶσα)        & ασα    & ασης   & αση     & ασαν \\
                          & ασαι   & ασων   & ασαις   & ασας \\
neut.~($\sim$ πᾶν)        & αν     & αντος  & αντι    & αν \\
                          & αντα   & αντων  & ασι(ν)  & αντα \\
\multicolumn{5}{l}{\textbf{athematic (e.g.~τιθείς, διδούς, δεικνύς)}}\\
masc. (nom., gen.)        &  \multicolumn{4}{l}{εις, εντος; ους, οντος; υς, υντος} \\
fem.                      &  \multicolumn{4}{l}{εισα, εισης; ουσα, ουσης; υσα, υσης} \\
neut.                     &  \multicolumn{4}{l}{εν, εντος; ον, οντος; υν, υντος} \\

\end{tabular}

\footnotetext[7]{also nouns, passive finite}
\footnotetext[8]{also comparatives, pl. gen.}

\pagebreak

Common irregular verbs:\\*
%
{ \footnotesize\setlength{\tabcolsep}{3pt}
%
\begin{tabular}{llllll}
\grammartablehdr{present} & \grammartablehdr{future} & \grammartablehdr{aorist} & \grammartablehdr{perfect} & \grammartablehdr{perf.~mid.} & \grammartablehdr{aor.~pass.} \\
ἄγω  &  ἄξω  &  ἤγαγον  &        &  ἤγμαι  &  ἤχθην        \\
αἰρέω  &  αἰρήσω  &  εἶλον  &  ᾔρηκα  &  ᾔρημαι  &  ᾐρέθην        \\
ἀκούω  &  ἀκούσω  &  ἤκουσα  &  ἀκήκοα  &        &  ἠκούσθην        \\
ἄρχω  & &&& ἄρξομαι  &  ἠρξάμην        \\
βαίνω  &  βήσω  &  ἔβησα  &  βέβηκα        \\
βάλλω  &  βαλέω  &  ἔβαλον  &  βέβλήκα  &  βεβλῆμαι  &  ἐβλήθην        \\
διώκω  &  διώξω  &  ἐδίωξα  &        &  δεδίωγμαι  &  ἐδιώχθην        \\
ἐσθίω  &  φάγομαι  &  ἔφαγον        \\
ἔχω  &  ἔξω  &  ἔσχον  &  ἔσχηκα        \\
ἐθέλω  &  ἐθελήσω  &  ἠθέλησα        \\
θνήσκω  &  θανοῦμαι  &  εθάνον  &  τεθνήκα        \\
καίω  &  καύσω  &  ἔκηα  &   &  κέκαυμαι  &  ἐκαύθην        \\
καλέω  &  καλέω  &  ἐκάλεσα  &  κέκληκα  &  κέκλημαι  &  ἐκλήθην        \\
κρίνω  &  κρινέω  &  ἔκρινα  &  κέκρικα  &  κέκριμαι  &  ἐκρίθην        \\
λαμβάνω  &  λάψομαι  &  ἔλαβον  &  εἴληφα  &  εἴλημμαι  &  ἐλήφθην        \\
λείπω  &  λείψω  &  ἔλιπον  &  λέλοιπα  &  λέλειμμαι  &  ἐλείφθην        \\
λύω  &  λύσω  &  ἔλυσα  &  λέλυκα  &  λέλυμαι  &  ἔλύθην        \\
μέλλω  &  μελλήσω  &  ἐμέλλησα        \\
μένω  &  μενέω  &  ἔμεινα  &  μεμένηκα        \\
μιμνήσκω  &   μνήσω  &  ἔμνησα  &  μέμνημαι  &     &  ἐμνήσθην        \\
ὁράω  &  ὄψομαι  &  εἶδον  &  ὄπωπα        \\
πείθω  &  πείσω  &  ἔπείσα  &  πέποίθα  &  πέπιεσμαι  &  ἐπείσθην        \\
πίνω  &  πίομαι  &  ἔπιον  &  πέπωκα  &  & ἐπόθην        \\
πίπτω  &  πεσέομαι  &  ἔπεσον  &  πέπτηκα        \\
τελείω  &  τελέσω  &  ἐτέλεσα  &  τετέλεκα  &  τετέλεσμαι  &  ἐτελέσθην        \\
φαίνω  &  φανέω  &  ἔφηνα  & & & ἐφάνην        \\
φέρω  &  οἴσω  &  ἤνεγκα  &  ἐνήνοχα  &    &  ἠνέχθην        \\
φεύγω  &  φεύξομαι  &  ἔφυγον  &  πέφευγα        \\
\end{tabular}

}

\end{small}

