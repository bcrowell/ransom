\pagebreak

\begin{small}

\newcommand{\grammartablehdr}[1]{{\textcolor{gray}{\emph{#1}}}}

\mychapter{Field guide to Homeric grammar}

\formatlikesection{Pronouns}

Homer has about 221 pronouns and articles (which are not distinct categories):

% Output of find_all_pronouns.rb and find_all_articles ; hoarded in Pronouns.all_no_grave

{\footnotesize 
ἅ αἱ αἵ αἵδε ἄμμε ἄμμες ἄμμι ἄμμιν ἅς ἄσσα ἅσσά ἅσσα ἑ ἕ ἐγώ ἔγωγέ ἔγωγε ἐγών ἑέ ἕης ἑθέν ἑθεν ἕθεν εἷο ἐμέ ἐμέθεν ἐμεῖο ἐμέο ἐμεῦ ἐμοί ἔμοιγε ἑο ἕο ἑοῖ εὑ ἥ ἡ ἥδέ ἥδε ἧμας ἡμέας ἡμεῖς ἡμείων ἡμέων ἡμῖν ἥμιν ἧμιν ἥν ἧς ᾗς κεῖνος μέ με μευ μίν μιν μοί μοι νώ νῶΐ νῶι νῶϊ νῶιν νῶϊν ὅ ὁ ὅδε οἱ οἵ οἷ οἵδε οἷσί οἷσι οἷσίν οἷσιν ὅν ὅου ὅς ὅστις ὁτέοισιν ὅτευ ὅτεῳ ὅτεῴ ὅτεών ὅτινα ὅτινας ὅτίς ὅτις ὅττεό ὅττευ οὗ οὕς σέ σε σέθεν σεῖο σέο σεο σευ σεῦ σοί σοι σοῦ σύ σφας σφε σφέας σφεας σφείων σφέων σφεων σφι σφιν σφίσι σφισι σφίσιν σφισιν σφώ σφωε σφῶΐ σφῶϊ σφωιν σφωϊν σφῶϊν σφῶν σφῷν τά τάδε ταί τάς τάσδε τάων τεΐν τέο τεο τεοῖο τευ τεῦ τεῳ τέων τῇ τῇδέ τήν τήνδε τῆς τῇς τῆσδέ τῆσδε τῇσι τῇσίν τῇσιν τί τι τίνα τινά τινα τινάς τινας τινε τίνες τινές τινες τινι τίς τις τό τόδε τοί τοι τοιάδε τοιαίδε τοιήδε τοῖιν τοῖϊν τοῖο τοιοίδε τοιόνδε τοιόσδε τοιούσδε τοῖς τοῖσδε τοίσδεσι τοίσδεσσι τοῖσδεσσι τοίσδεσσιν τοῖσί τοῖσι τοῖσίν τοῖσιν τόν τόνδε τοσόνδέ τοσόνδε τοσσάδε τοσσόνδε τοῦ τοῦδέ τοῦδε τούς τούσδε τύνη τώ τῳ τῷ τώδε τῷδε τῶν τῶνδε ὑμέας ὑμεῖς ὑμείων ὑμῖν ὕμιν ὔμμε ὔμμες ὔμμι ὔμμιν χἠμεῖς ὥ ᾧ ὧν
}

They proliferate because (1) Homer mixes Aeolic and Ionian words, (2) some pronouns come in both emphatic and unemphatic forms,
(3) sometimes there are contractions of ε, and (4) there are suffixes -δε (here) and -θεν (genitive, from).

\formatlikesubsection{Personal pronouns}

The most common Ionian personal pronouns are:\\
%
\begin{tabular}{lll}
N & ἐγώ σύ εἷο     & ἡμεῖς ὑμεῖς -- \\
G & ἐμεῖο σεῖο εἷο & ἡμείων ὑμείων σφείων \\
D & ἐμοί σοί ἑοί   & ἡμῖν ὑμῖν σφίσι \\
A & ἐμέ σέ ἑέ      & ἡμέας ὑμέας σφέας
\end{tabular}\\
%
These forms are used for emphasis and with prepositions.
Contractions happen mainly in the genitive. They take -εῖο to -έο and -εῦ (both occur), and -είων to -έων.
A few other contractions exist, such as ἕ=ἑέ and σφάς=σφέας. Τεΐν=σοί. 

\pagebreak

The older Aeolic forms that differ are:\\
%
\begin{tabular}{lll}
N & -- -- --  & ἄμμες ὔμμες -- \\
G & ἔμεθεν σέθεν ἕθεν & -- -- -- \\
D & -- -- --  & ἄμμιν ὔμμιν -- \\
A & -- -- --  & ἄμμε ὔμμε -- 
\end{tabular}

The third-person pronouns, where they exist, are actually not personal but
rather refer to other words or phrases, although like the true personal
pronouns they are not inflected for gender. Sometimes they are used as reflexives.
They are uncommon in Homer, and more
frequently he uses forms of ὁ, ἡ, τό, which can be used for this purpose as well
as being demonstrative and relative pronouns (see below). Example: τὴν δ᾽ ἐγὼ οὐ λύσω,
``but I will not release her'' (Iliad 1.29).

The unemphatic forms are:\\
%
\begin{tabular}{lll}
G & μευ σεο+σευ ἑο+ἑυ & -- -- σφεων \\
D & μοι τοι οἱ        & -- -- σφισι \\
A & με σε ἑ+μιν       & -- -- σφεας
\end{tabular}\\
%
These are enclitic. The distinction between emphatic and unemphatic pronouns
is usually reinforced by word order:
``δοκεῖ μοι,'' but ``ἐμοὶ δοκεῖ.'' As in English, the pronoun's normal position is after
the verb (the dog bit me), and fronting it is for emphasis (it's me
that the dog bit). For stronger emphasis, -γε can be added: ἔγωγε.
% Opposite of French, e.g., dites moi, me dire, because in french normal position is before verb.
% But not always?
% See Dik, ``On Unemphatic `Emphatic' Pronouns in Greek,'' sci-hub.se/10.2307/4433482
% Her example: Μή σε, γέρον κοίλῃσιν ἐγὼ παρὰ νηυσὶ κιχείω (Iliad 1.26).

Duals are formed with νῶι- (1p) and σφωι- (2+3p), and end in -ν for the genitive and dative.

\formatlikesubsection{Possession:}

The language in general uses genitive and dative
nouns to modify other nouns, σκῆπτρον θεοῖο (the scepter of the god, genitive), and this is commonly done with
pronouns to show possession. Dative pronouns are often used: μοι αἶσα (my fate), τοι αἷμα (your blood), οἱ ἦτορ (his heart).

There are also adjectival pronouns (Smyth 330): my ἐμός, our ἡμέτερος/ἁμός,
your (s.) σός/τεός/ὑμός, your (pl.) ὑμέτερος. These agree in number, case, and gender with the noun
being modified: νῆας ἐμάς (my ships). They can also mean my own, mine, etc.

The third-person adjectival pronouns generally mean his
own, her own, and their own, rather than simply his, her, or their. The singular forms are
commonly just the forms of ὅ (see below). There are also singular ἑός and plural σφός/σφέτερος.


% μοι αἶσα = Iliad 24.237
% αἶψά τοι αἷμα κελαινὸν ἐρωήσει περὶ δουρί. = Iliad 1.303
% οἱ ἦτορ = Iliad 1.188
% These are what Smyth refers to as the dative of the possessor (p. 341, sec 1480) and Benner calls a dative of interest.
% νῆας ἐμάς = Iliad 9.361

\formatlikesubsection{More about the forms of ὁ/ὅ:}

The accented word ὅ in its various forms serves as a relative pronoun:

Μῆνιν \ldots ἣ μυρί ̓ Ἀχαιοῖς ἄλγε ̓ ἔθηκε\\
Rage \ldots that caused the Achaeans many woes (Iliad 1.2)

The unaccented ὁ has two functions. Demonstrative noun or adjective: --

τὰ πένοντο\\
they were busy with these things (Iliad 1.313)

Third-person personal pronoun (a more common usage than εἷο, etc.): --

ὁ γὰρ βασιλῆϊ χολωθεὶς\\
for he, enraged with the king (Iliad 1.9)

(In later dialects, ὁ became the definite article.) The accents make the accented forms stand out both in speech
and in print, so that they form big landmarks breaking up a sentence into clauses. The wimpy unaccented forms
don't stand out and lean on the following word (are proclitic). For more detail, see Smyth 338b, 1105, 1113.

Declension:

% The following is based on my report_inflections.rb script:

\begin{tabular}{lll}
m: & ὅ τοῖο/τοῦ τῷ/οἱ τόν & οἵ/τοί τῶν τοῖσ(ιν) τούς \\
f: & ἥ τῆς τῇ τήν & αἵ/ταί τάων/τῶν τῇσ(ιν) τάς \\
n: & τό τοῖο/τοῦ τῷ τό/ὅττι & τά τῶν τοῖσ(ιν) τά \\
\end{tabular}

The word τῷ also functions as an adverb meaning therefore or in that case.

\formatlikesubsection{Demonstratives; suffixes -δε and -θεν}

The demonstratives are ὁ, ὅδε, οὗτος, ἐκεῖνος. They function as nouns and adjectives. 

The suffix -δε, cognate with English ``to,'' means ``here,'' so
that ὅδε means ``this'' as opposed to ``that.''

The suffix -θεν is a fossilized remnant of the ablative case, indicating
separation or motion away from something. It forms adverbs like οὐρανόθεν, from heaven.
Because the ablative was absorbed into
the genitive case, a form like σέθεν functions as an alternative for the genitive σεῖο.
This can produce usages in which ``from'' seems paradoxical to English speakers. For example, ξυνίημι + gen.~
means to listen to someone, so that ξυνίημι σέθεν means that I listen to you.

\formatlikesubsection{Correlatives}

\begin{tabular}{llllll}
& \grammartablehdr{interrog.} 	& \grammartablehdr{some+X} & 	\grammartablehdr{demonstr.} & 	\grammartablehdr{rel.} & 	\grammartablehdr{X+ever} \\
& τίς &    τις &     &           ὅς &     ὅστις  \\
& ποῦ &    που &         &        οὗ &     ὅπου \\
& πότε &   ποτέ &    τότε &       ὅτε &    ὁπότε \\
& πῶς &    πως &     οὕτως &      ὡς &     ὅπως \\
& ποῖος &  ποιός &   τοιοῦτος &   οἷος &   ὁποῖος \\
& πόσος &  ποσός &   τοσοῦτος &   ὅσος &   ὁπόσος
\end{tabular}

``Why?'' can be expressed using τί (a neuter accusative form of τις) or τί ἤ.

% τί ἤ can also be spelled τίη (Cunliffe)
% Τέκνον, τί κλαίεις; (Iliad 1.362) τί ἤ τοι ταῦτα ἰδυίῃ πάντ᾽ ἀγορεύω; (1.365)

\vfill\pagebreak

\formatlikesection{Itty bitties}

What I mean by an ``itty bitty'' is a short word that contributes disproportionately to confusion.
Many of the these words are two or three letters, and many are among the ten or twenty the most common
words in the Homeric dialect. 

A \emph{particle} is a word that has no meaning of its own and changes the meaning of other words.
A \emph{clitic} is a word that gets controlled phonologically by other words, leaning (κλίνω) on them. In Greek,
clitics usually lack an accent (see Smyth 182 for details), although they may be listed with one in dictionary entries.
Clitics consist of \emph{proclitics} that lean on the following word, and
\emph{enclitics} that lean on the preceding word. Articles and prepositions are enclitics.
A \emph{postpositive} is a word that comes after the word that it modifies, as in ``someone nice.''

\formatlikesubsection{Particles}

Homer uses the following 39 particles:

% Output of find_all_particles.rb
% Omitted the following, which are not listed in Cunliffe, are probably just weird respellings:
% πεῤ πη πῆ ἄῤ 
% Added οὔτε, which seems analogous to μήτε, but which perseus doesn't analyze as a separate particle?
% Added οὐκ/οὐχ, which perseus analyzes as forms of the lemma οὐ.

ἄν ἄρ ἄρα ἀτάρ αὐτάρ γάρ γε δαί δέ δή ἦ ἤτοι ἠΰτε θην κε κεν μά μάν μέν μη μήν μήτε οὐ οὐκ οὖν οὐχ οὔτε περ πῃ πῇ ποθεν ποτε πω πως ῥα τάρ τε τοι τοιγάρ

There is usually no difference in meaning between accented and unaccented forms except that an accent can indicate emphasis.
For a systematic discussion, see Smyth, sec.~2769, p. 631, or Monro, ch.~13.

Particles sorted according to their function:

\begin{tabular}{p{1.6in}p{2.7in}}
affirmative, emphasis, \mbox{interrogative}, oaths &
  γε δαί ἦ/ἤτοι θην μά μάν μήν πω τάρ τοι \tabularnewline \hline
negation, prohibition &
  μη οὐ/οὐκ/οὐχ \tabularnewline \hline
coordination, correlatives &
  μήτε οὔτε τε \tabularnewline \hline
and/but, adversative, \mbox{continuation} &
  ἀτάρ αὐτάρ \tabularnewline \hline
temporal order, causation, conjunction &
  ἄρ/ἄρα/ῥα/ἄῤ γάρ δέ δή τοιγάρ \tabularnewline \hline
when, where, how &
  πῃ πῇ ποθεν ποτε πως \tabularnewline \hline
potential, conditional, \mbox{counterfactual}, iteration &
  ἄν κε/κεν \tabularnewline \hline
similes &
  ἠΰτε \tabularnewline \hline
multiple uses &
  μέν οὖν περ
\end{tabular}

\formatlikesubsection{Conjunctions}

καί - and\\
ἀλλά - but\\
γάρ - for; postpositive\\
δέ - and, but (not a negative like modern δεν)\\
ἐάν/ἤν - εἰ + ἄν (if+subjunctive particle)
ὅμως - nevertheless\\

\formatlikesubsection{Negation}

οὐ(κ), οὐχ - not; proclitic\\
μή - negative form used in imperatives\\

\formatlikesubsection{List of itty bitties}


κε(ν) - used like ``if'' to limit verbs\\
αν - like κε; in Homer, may be more emphatic or used more often for negative clauses\\
εἰ - ``if;'' can be used, e.g., as εἰ κεν + verb; proclitic\\
αἰ - Aeolic form of εἰ; may imply a wish or purpose\\
αἰ κε(ν) - if only, so that\\
αἲ γάρ - oh, that \ldots !\\
ἤν = εἰ ἄν; also an interjection, ``see there!;'' cf.~epic pronoun ἥν\\
ἄρ(α)/ῥα - time or causation: then/next, therefore; postpositive; in later dialects, can introduce
      a question, as in ``Who, then, will fight?''\\
γάρ = γε ἄρ ``for;'' postpositive\\
δέ - but, and, or supplying the reason for something; cf. postposition -δε, ``to''\\
αὐτάρ, ἀτάρ - similar to δέ, poetic\\
τε - correlative/connecting particle; always postpositive?; enclitic\\
μέν, μήν - affirmative particles; the difference depends on prose/verse and meter\\
τοι - (1) synonym for dative pronoun σοι; (2) affirmative particle; both enclitic\\

\formatlikesubsection{Oaths, emphasis, emphatics, and doubt}

γε - used before or after a word to mark or emphasize it; often ``at least;'' postpositive, enclitic\\
δή - indeed, truly; postpositive\\
ἦ μέν - used in oaths\\
ἤτοι - indeed, truly (also used in either/or constructions)\\
πού - indicates doubt or supposition, ``I guess,'' or ``no doubt;'' (also somewhere, anywhere, somehow)\\
ὀΐω - finite verb used in speech to mean ``I think,'' or ``I believe''\\
ἄγε - used in speech to emphasize a command (lit.~the impv.~of ἄγω)

\formatlikesubsection{Time, causation, and temporal order}

ἤδη - already, now\\
νῦ(ν) - adverb; now, just now, presently; cf.~enclitic νυν (rare in Homer)\\
οὖν - postpositive adverb; so, then\\
ἄρ(α)/ῥα - so, then, after all\\
ὥστε - so that; adverb+inf; conjuction\\
ἅμα - at the same time with, together with\\
εὖτε - when, as, since\\
τέως/ἕως - meanwhile, for a time\\
ἵνα - so that, with subj., opt.; (other meanings related to place)\\
ὄφρα - so that, with subj., opt.; (other meanings related to time)

\formatlikesubsection{Correlatives}

In later dialects, where articles are common, one often has
the postpositive between the article and the noun, e.g.,
ὁ τ᾽ ἥλιος καὶ τὴν σελήνη.

τε \ldots και - A τε \ldots και Β, both A and B\\
μέν \ldots δέ \ldots - contrasting, ``and on the other hand''\\
οὔτε \ldots οὔτε - neither \ldots nor\\

\vfill

\pagebreak

\formatlikesection{Verbs}

\newcommand{\tca}{\cellcolor{TableColorA}}
\newcommand{\tcb}{\cellcolor{TableColorB}}

Infixes in verbs:\\*
%
\begin{tabular}{llll}
root aor.         & η/ω/υ & passive aor.      & σθη \\
future              & σ     & future passive      & θεσ \\
perfect active      & κ     & future middle perfect & κσ \\
present, 2nd aor. opt.    & οι(η) (αοι->ῳ) & participle        & ντ\footnotemark \\
aor. optative     & \multicolumn{3}{l}{(σ)αι(η)/(σ)ει(η)}\\
\end{tabular}

Personal endings:\\*
%
\begin{tabular}{lllllll}
\textbf{active, -ω verbs}\\
present, future              & ω      & εις\footnotemark  & ει\footnotemark[3]  & ομεν    & ετε    & ουσι(ν) \\
impf.; them. 2nd aor.        & ον\footnotemark[4] & ες      & ε(ν)      & ομεν    &	ετε    & ον\footnotemark[4] \\
aorist                       & α      & ας      & ε(ν)      & αμεν    & ατε    & αν\footnotemark[5] \\
root aorist (-η/ω/υ-)        & \tca{}ν & \tca{}ς  & \tca{}-  & \tca{}μεν & \tca{}τε & \tca{}σαν \\
\textbf{active, -μι verbs}\\
present                      & μι     & ς       & σι        & μεν     & τε     & ασι \\
aorist, impf.                & \tca{}ν & \tca{}ς  & \tca{}-  & \tca{}μεν & \tca{}τε & \tca{}σαν \\
\textbf{middle}\\
present                      & \tcb{}μαι & εαι  & \tcb{}ται & \tcb{}μεθα & \tcb{}σθε & \tcb{}νται \\
impf.; aor. ind., opt.       & μην    & σο      & το        & μεθα    &	σθε    & ντο/ατο\\
\textbf{passive}\\
fut; pres., athematic        & \tcb{}μαι  & αι  & \tcb{}ται & \tcb{}μεθα & \tcb{}σθε & \tcb{}νται \\
\ldots thematic              & ομαι   & ει\footnotemark[3] & εται      & ομεθα   & εσθε   & ονται \\
aor. (1st θη-, 2d η-)        & \tca{}ν & \tca{}ς  & \tca{}-/το  & \tca{}μεν & \tca{}τε & \tca{}σαν/θεν/ντο \\
\textbf{optative}\\
present (-αι/ει/οι-)         & μι/ην  & ς/ης    & -/η/οῖ    & μεν     & τε     & εν \\
aorist (\ldots)              & μι     & ς       & -         & μεν     & τε     & εν \\
aorist, alt. forms           &        & ας      & ε(ν)      &         &        & αν \\
\textbf{imperative}\\
active                       &  \multicolumn{3}{l}{pres.~ε/θι/τι\footnotemark[6], aor.~ον}   & \multicolumn{3}{l}{pres.~ετε, aor.~ατε }      \\
mediopassive                 &  \multicolumn{3}{l}{pres.~σο/ου, aor.~αι/θητι}                & \multicolumn{3}{l}{pres.~εσθε, aor.~ασθε/θητε }     \\
\end{tabular}

\footnotetext[1]{also 3 pl mp}
\footnotetext[2]{also participle}
\footnotetext[3]{both active 3 and pass 2}
\footnotetext[4]{both 1 sing and 3 pl}
\footnotetext[5]{both aorist and some participles}
\footnotetext[6]{θι is omitted in present, and becomes τι after another aspirated consonant}

\pagebreak

% Smyth, sec. 469
% https://latin.stackexchange.com/a/16856/3597

Infinitives:\\*
%
\begin{tabular}{p{1.25in}p{2.75in}}
-εν/ειν, -μεναι, -μεν   &  present, thematic 2nd aor active, fut active \\
-αι\footnotemark[7]    &  1st aor active \\
-ναι, -μεναι, -μεν  & present and 2nd perfect of athematic verbs; passive aor; perfect active \\
-μεναι              &  athematic 2nd aor \\
-σθαι               &  mediopassive, except aor passive \\
\end{tabular}

Participles:\\*
%
\begin{tabular}{lllll}
                          & \multicolumn{4}{l}{[\emph{-αντο is middle impf., never a participle.}]} \\
\multicolumn{5}{l}{\textbf{present, 2nd aorist}}\\
masc.~($\sim$ γέρων)      & ων\footnotemark[8]/ους & οντος  & οντι  & οντα      \\
                          & οντες  & οντων & ουσι(ν) & οντας \\
fem.~($\sim$ θάλασσα)     & ουσα   & ουσης & ούσῃ & ουσαν \\
                          & ουσαι  & ουσάων & ούσῃς & ουσας \\
neut.                     & ον     & οντος  & οντι  & ον \\
                          & οντα   & οντων  & ουσι(ν) & οντα \\
\multicolumn{5}{l}{\textbf{middle}}\\
                          & μενος \\
\multicolumn{5}{l}{\textbf{1st aorist}}\\
masc.~($\sim$ πᾶς)        & ας     & αντος  &αντι     & αντα \\
                          & αντες  & αντων  & ασι(ν)  & αντας \\
fem.~($\sim$ πᾶσα)        & ασα    & ασης   & αση     & ασαν \\
                          & ασαι   & ασων   & ασαις   & ασας \\
neut.~($\sim$ πᾶν)        & αν     & αντος  & αντι    & αν \\
                          & αντα   & αντων  & ασι(ν)  & αντα \\
\multicolumn{5}{l}{\textbf{athematic (e.g.~τιθείς, διδούς, δεικνύς)}}\\
masc. (nom., gen.)        &  \multicolumn{4}{l}{εις, εντος; ους, οντος; υς, υντος} \\
fem.                      &  \multicolumn{4}{l}{εισα, εισης; ουσα, ουσης; υσα, υσης} \\
neut.                     &  \multicolumn{4}{l}{εν, εντος; ον, οντος; υν, υντος} \\

\end{tabular}

\footnotetext[7]{also nouns, passive finite}
\footnotetext[8]{also comparatives, pl. gen.}

\pagebreak


Common irregular verbs:\\*
%
{ \footnotesize\setlength{\tabcolsep}{3pt}
%
\begin{tabular}{llllll}
\grammartablehdr{present} & \grammartablehdr{future} & \grammartablehdr{aorist} & \grammartablehdr{perfect} & \grammartablehdr{perf.~mid.} & \grammartablehdr{aor.~pass.} \\
ἄγω  &  ἄξω  &  ἤγαγον  &        &  ἤγμαι  &  ἤχθην        \\
αἰρέω  &  αἰρήσω  &  εἶλον  &  ᾔρηκα  &  ᾔρημαι  &  ᾐρέθην        \\
ἀκούω  &  ἀκούσω  &  ἤκουσα  &  ἀκήκοα  &        &  ἠκούσθην        \\
ἄρχω  & &&& ἄρξομαι  &  ἠρξάμην        \\
βαίνω  &  βήσω  &  ἔβησα  &  βέβηκα        \\
βάλλω  &  βαλέω  &  ἔβαλον  &  βέβλήκα  &  βεβλῆμαι  &  ἐβλήθην        \\
διώκω  &  διώξω  &  ἐδίωξα  &        &  δεδίωγμαι  &  ἐδιώχθην        \\
ἐσθίω  &  φάγομαι  &  ἔφαγον        \\
ἔχω  &  ἔξω  &  ἔσχον  &  ἔσχηκα        \\
ἐθέλω  &  ἐθελήσω  &  ἠθέλησα        \\
θνήσκω  &  θανοῦμαι  &  εθάνον  &  τεθνήκα        \\
καίω  &  καύσω  &  ἔκηα  &   &  κέκαυμαι  &  ἐκαύθην        \\
καλέω  &  καλέω  &  ἐκάλεσα  &  κέκληκα  &  κέκλημαι  &  ἐκλήθην        \\
κρίνω  &  κρινέω  &  ἔκρινα  &  κέκρικα  &  κέκριμαι  &  ἐκρίθην        \\
λαμβάνω  &  λάψομαι  &  ἔλαβον  &  εἴληφα  &  εἴλημμαι  &  ἐλήφθην        \\
λείπω  &  λείψω  &  ἔλιπον  &  λέλοιπα  &  λέλειμμαι  &  ἐλείφθην        \\
λύω  &  λύσω  &  ἔλυσα  &  λέλυκα  &  λέλυμαι  &  ἔλύθην        \\
μέλλω  &  μελλήσω  &  ἐμέλλησα        \\
μένω  &  μενέω  &  ἔμεινα  &  μεμένηκα        \\
μιμνήσκω  &   μνήσω  &  ἔμνησα  &  μέμνημαι  &     &  ἐμνήσθην        \\
ὁράω  &  ὄψομαι  &  εἶδον  &  ὄπωπα        \\
πείθω  &  πείσω  &  ἔπείσα  &  πέποίθα  &  πέπιεσμαι  &  ἐπείσθην        \\
πίνω  &  πίομαι  &  ἔπιον  &  πέπωκα  &  & ἐπόθην        \\
πίπτω  &  πεσέομαι  &  ἔπεσον  &  πέπτηκα        \\
τελείω  &  τελέσω  &  ἐτέλεσα  &  τετέλεκα  &  τετέλεσμαι  &  ἐτελέσθην        \\
φαίνω  &  φανέω  &  ἔφηνα  & & & ἐφάνην        \\
φέρω  &  οἴσω  &  ἤνεγκα  &  ἐνήνοχα  &    &  ἠνέχθην        \\
φεύγω  &  φεύξομαι  &  ἔφυγον  &  πέφευγα        \\
\end{tabular}

}

\end{small}

