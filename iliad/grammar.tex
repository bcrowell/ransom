\pagebreak

\begin{small}

\newcommand{\grammartablehdr}[1]{{\textcolor{gray}{\emph{#1}}}}

\mychapter{Field guide to Homeric grammar}

\formatlikesection{Itty bitties}

What I mean by an ``itty bitty'' is a short word that contributes disproportionately to confusion.
Many of the following words are two or three letters, and many are among the ten or twenty the most common
words in the Homeric dialect. Many are verbal particles, clitics, or postpositives.

A \emph{particle} is a word that has no meaning of its own and changes the meaning of other words.
A \emph{clitic} is a word that gets controlled phonologically by other words, leaning (κλίνω) on them. In Greek,
clitics lack an accent, although they are listed with one in dictionary entries. 
They consist of \emph{proclitics} that lean on the following word, and
\emph{enclitics} that lean on the preceding word. Articles and prepositions are enclitics.
A \emph{postpositive} is a word that comes after the word that it modifies, as in ``someone nice.''

\formatlikesubsection{Conjunctions}

καί - and\\
ἀλλά - but\\
γάρ - for; postpositive\\
δέ - and, but (not a negative like modern δεν)\\
ὅμως - nevertheless\\

\formatlikesubsection{Negation}

οὐ(κ), οὐχ - not; proclitic\\
μή - negative form used in imperatives\\

\formatlikesubsection{List of itty bitties}


κε(ν) - used like ``if'' to limit verbs\\
αν - like κε, but used in Homer for negative clauses\\
εἰ - ``if;'' can be used, e.g., as εἰ κεν + verb; proclitic\\
αἰ - if; implies a wish or purpose\\
αἰ κε(ν) - if only, so that\\
αἲ γάρ - oh, that \ldots !\\
ἣν = εἰ ἄν; also an interjection, ``see there!;'' cf. epic pronoun ἥν
γε - used before or after a word to mark or emphasize it; often ``at least;'' postpositive, enclitic\\
ἄρ(α)/ῥα - time or causation: then/next, therefore; postpositive; in later dialects, can introduce
      a question, as in ``Who, then, will fight?''\\
γάρ = γε ἄρ ``for;'' postpositive\\
δέ - but, and, or supplying the reason for something; cf. postposition -δε, ``to''\\
αὐτάρ, ἀτάρ - similar to δέ\\
τε - correlative/connecting particle; always postpositive?; enclitic\\
μέν, μήν - affirmative particles; the difference depends on prose/verse and meter\\

\formatlikesubsection{Oaths, emphasis, and emphatics}

δή - indeed, truly; postpositive\\
ἦ μέν - used in oaths\\
ἤτοι - indeed, truly (also used in either/or constructions)\\

\formatlikesubsection{Time, causation, and temporal order}

ἤδη - already, now\\
νῦν - now, just now, presently; enclitic\\
οὖν - postpositive adverb; so, then\\
ἄρ(α)/ῥα - so, then\\
ὥστε - so that; adverb+inf; conjuction\\

\formatlikesubsection{Correlatives}

In later dialects, where articles are common, one often has
the postpositive between the article and the noun, e.g.,
ὁ τ᾽ ἥλιος καὶ τὴν σελήνη.

τε \ldots και - A τε \ldots και Β, both A and B\\
μέν \ldots δέ \ldots - contrasting, ``and on the other hand''\\
οὔτε \ldots οὔτε - neither \ldots nor\\

\formatlikesection{Correlatives}

\begin{tabular}{llllll}
& \grammartablehdr{interrog.} 	& \grammartablehdr{some+X} & 	\grammartablehdr{demonstr.} & 	\grammartablehdr{rel.} & 	\grammartablehdr{X+ever} \\
& τίς &    τις &     &           ὅς &     ὅστις  \\
& ποῦ &    που &         &        οὗ &     ὅπου \\
& πότε &   ποτέ &    τότε &       ὅτε &    ὁπότε \\
& πῶς &    πως &     οὕτως &      ὡς &     ὅπως \\
& ποῖος &  ποιός &   τοιοῦτος &   οἷος &   ὁποῖος \\
& πόσος &  ποσός &   τοσοῦτος &   ὅσος &   ὁπόσος
\end{tabular}

\end{small}

\pagebreak
